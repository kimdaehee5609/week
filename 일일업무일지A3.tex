%	-------------------------------------------------------------------------------
%
%		작성 : 2021.
%				12.
%				14	
%
%
%
%
%
%	-------------------------------------------------------------------------------

%\documentclass[10pt,xcolor=pdftex,dvipsnames,table]{beamer}
%\documentclass[10pt,blue,xcolor=pdftex,dvipsnames,table,handout]{beamer}
%\documentclass[14pt,blue,xcolor=pdftex,dvipsnames,table,handout]{beamer}
\documentclass[aspectratio=1610,17pt,xcolor=pdftex,dvipsnames,table,handout]{beamer}


	%	========================================================== 	Package
		\usepackage{kotex}						% 한글 사용
		\usepackage{amssymb,amsfonts,amsmath}	% 수학 수식 사용
		\usepackage{color}					%
		\usepackage{colortbl}					%


	%		========================================================= 	note 옵션인 
	%			\setbeameroption{show only notes}
		

	%		========================================================= 	Theme

		%	---------------------------------------------------------	
		%	전체 테마
		%	---------------------------------------------------------	
		%	테마 명명의 관례 : 도시 이름
%			\usetheme{default}			%
%			\usetheme{Madrid}    		%
%			\usetheme{CambridgeUS}    	% -red, no navigation bar
%			\usetheme{Antibes}			% -blueish, tree-like navigation bar

		%	----------------- table of contents in sidebar
			\usetheme{Berkeley}		% -blueish, table of contents in sidebar
									% 개인적으로 마음에 듬



		%	---------------------------------------------------------	
		%	Font Theme 				폰트 테마
		%	---------------------------------------------------------	
%			\usfonttheme{default}		
			\usefonttheme{serif}			
%			\usefonttheme{structurebold}			
%			\usefonttheme{structureitalicserif}			
%			\usefonttheme{structuresmallcapsserif}			



		%	---------------------------------------------------------	
		%	Inner Theme 				
		%	---------------------------------------------------------	

%			\useinnertheme{default}
			\useinnertheme{circles}		% 원문자			
%			\useinnertheme{rectangles}		% 사각문자			
%			\useinnertheme{rounded}			% 깨어짐
%			\useinnertheme{inmargin}			




		\setbeamercovered{dynamic}





		% --------------------------------- 	문서 기본 사항 설정
		\setcounter{secnumdepth}{3} 		% 문단 번호 깊이
		\setcounter{tocdepth}{3} 			% 문단 번호 깊이




% ------------------------------------------------------------------------------
% Begin document (Content goes below)
% ------------------------------------------------------------------------------
	\begin{document}
	

			\title{김대희 메모장}

			\author{김대희}

			\date{ 	2022년 
					8월 
					24일 
					수요일 }


% -----------------------------------------------------------------------------
%		개정 내용
% -----------------------------------------------------------------------------
%
%			2020.10.31 일 작성
%
%
%
%
%
%
%
	%	==========================================================
	% 표지  
	%	---------------------------------------------------------- page 01
		\begin{frame}[plain]
		\titlepage
		\end{frame}

		\clearpage



	%	----------------------------------------------------------  
	%		Frame   
	%	---------------------------------------------------------- page 02
		\begin{frame} [t,plain]

			\begin{block} { page 02  :  업무 }

			\end{block}			
								
		\end{frame}	 % ---------------------------------------------					 




	%	----------------------------------------------------------  
	%		Frame   
	%	---------------------------------------------------------- page 03
		\begin{frame} [t,plain]

			\begin{block} { page 03  :  개인 }

			\end{block}			
								
		\end{frame}	 % ---------------------------------------------					 



	%	----------------------------------------------------------  
	%		Frame   
	%	---------------------------------------------------------- page 04
		\begin{frame} [t,plain]

			\begin{block} { page 04  :  점심 }

			\end{block}			
								
		\end{frame}	 % ---------------------------------------------					 




	%	----------------------------------------------------------  아침에 출력
	%		Frame   
	%	---------------------------------------------------------- page 05
		\begin{frame} [t,plain]

			\begin{block} { page 05 }

			\begin{itemize}
			\item 	마음 챙김
			\item 	지적 대화
			\item 	수학
			\item 	심리수업
			\item 	마음 챙김
%			\item 	지식 수업
			\item 	데일리
			\item  	Mat plot lib 
			\end{itemize}


			\end{block}			
								
		\end{frame}	 % ---------------------------------------------					 



	%	----------------------------------------------------------  
	%		Frame   
	%	---------------------------------------------------------- page 06
		\begin{frame} [t,plain]

			\begin{block} { page 06  :  점심때  }

			\end{block}			
								
		\end{frame}	 % ---------------------------------------------					 


	%	----------------------------------------------------------  
	%		Frame   
	%	---------------------------------------------------------- page 07
		\begin{frame} [t,plain]

			\begin{block} { page 07  :   저녁  }

			\end{block}			
								
		\end{frame}	 % ---------------------------------------------					 


	%	----------------------------------------------------------  
	%		Frame   
	%	---------------------------------------------------------- page 08
		\begin{frame} [t,plain]

			\begin{block} { page 08 }

			\end{block}			
								
		\end{frame}	 % ---------------------------------------------					 


	%	----------------------------------------------------------  
	%		Frame   
	%	---------------------------------------------------------- page 09
		\begin{frame} [t,plain]

			\begin{block} { page 09 }

			\end{block}			
								
		\end{frame}	 % ---------------------------------------------					 






				

	%	----------------------------------------------------------  
	%		Frame   
	%	---------------------------------------------------------- page 10
		\begin{frame} [t,plain]

			\begin{block} { page 10 }

			\end{block}			
								
		\end{frame}	 % ---------------------------------------------					 


	%	----------------------------------------------------------  
	%		Frame   
	%	---------------------------------------------------------- page 11
		\begin{frame} [t,plain]

			\begin{block} { page 11 }

			\end{block}			
								
		\end{frame}	 % ---------------------------------------------					 


	%	----------------------------------------------------------  
	%		Frame   
	%	---------------------------------------------------------- page 12
		\begin{frame} [t,plain]

			\begin{block} { page 12 }

			\end{block}			
								
		\end{frame}	 % ---------------------------------------------					 



	%	----------------------------------------------------------   		도서검색
	%		Frame   
	%	---------------------------------------------------------- page 13
		\begin{frame} [t,plain]

			\begin{block} { page 13  :  도서 검색 }

			\end{block}			
								
		\end{frame}	 % ---------------------------------------------					 



	%	----------------------------------------------------------  		
	%		Frame   
	%	---------------------------------------------------------- page 14
		\begin{frame} [t,plain]

			\begin{block} { page 14 }

			\end{block}			
								
		\end{frame}	 % ---------------------------------------------					 


	%	----------------------------------------------------------  
	%		Frame   
	%	---------------------------------------------------------- page 15
		\begin{frame} [t,plain]

			\begin{block} { page 15 : PX }

			\begin{itemize}
			\item 	발렌타인 17년 : 68,600
			\item 	파스칼VII (X.O Gold) : 48,710
			\end{itemize}


%			\begin{itemize}
%			\item 	송가인 달팽이 크림 : 5개
%			\item 	갑장 맥주 1박스 
%			\end{itemize}

			\end{block}			
								
		\end{frame}	 % ---------------------------------------------					 


	%	----------------------------------------------------------   주간일정
	%		Frame   
	%	---------------------------------------------------------- page 16
		\begin{frame} [t,plain]

			\begin{block} { page 16 : 주간 일정}
			\begin{itemize}
			\item 	8. 15. 월 234 : 광복절
			\item 	8. 16. 화 235 : 천수경 수업
			\item 	8. 17. 수 236 : 
			\item 	8. 18. 목 237 : 
			\item 	8. 19. 금 238 : 두드림 수업
			\item 	8. 20. 토 239 : 이발
			\item 	8. 21. 일 240 : 금성고3-4 야유회
			\end{itemize}
			\end{block}			
								
		\end{frame}	 % ---------------------------------------------					 

% 매주 화요일 수업 수요일 치과 치료


	%	----------------------------------------------------------  
	%		Frame   
	%	---------------------------------------------------------- page 17	아침약
		\begin{frame} [t,plain]


			\begin{block} { 아침약 : 메리놀 병원 - 소망약국 (비뇨기과)  }
			\begin{itemize}
			\item 	빨간색 알약 :  일양 하이드린정 : 혈관 확장 (서랍장)
			\end{itemize}
			\end{block}		

			\begin{block} { 아침약 : 메리놀병원 - 제일약국 (이비인후과) }
			\begin{itemize}
			\item 	[1] 케이컵정 50mg  위산분비억제제
			\item 	[2] 알레스틴 정   항알레르기약
			\item 	[3] 에르도스 캅셀 300 mg 객담 배출
			\item 	[4] 아세펙트캡슐 기관지 확장제
			\end{itemize}
			\end{block}			

	
								
		\end{frame}	 % ---------------------------------------------					 


	%	----------------------------------------------------------  
	%		Frame   
	%	---------------------------------------------------------- page 18	점심약
		\begin{frame} [t,plain]

			\begin{block} { 점심약 : 동아대병원 - 동대제일약국 }
			\begin{itemize}
			\item 	힌색 길쭉한 알약 : 글루파 500mg : 혈당 강하제(당뇨)
			\end{itemize}
			\end{block}			

			\begin{block} { 저녁  :  건강보조제 }
			\begin{itemize}
			\item 	초유 푸로틴 : 단백질 
			\item 	눈에 좋은 루테인
			\item 	고려은단 비타민C 1000
			\item 	우루사 100mg - 강은철
			\end{itemize}
			\end{block}			

								
		\end{frame}	 % ---------------------------------------------					 


	%	----------------------------------------------------------  
	%		Frame   
	%	---------------------------------------------------------- page 19	저녁약
		\begin{frame} [t,plain]

			\begin{block} { 저녁약 : 동아대병원 -  동대제일약국 (뇌혈관)}
			\begin{itemize}
			\item 	크레스토정10밀리그램(로수바스타틴칼슘)  (분황색 원형 작은것) 고지혈증 치료
			\item 	플라빅스정75밀리그람(클로피도그렐황산염)  (분홍색 원형 큰것) 혈전 생성 억제제
			\item 	애니디핀정 5mg () 혈관확장
			\item 	글루파정500mg(매트포르민염산염) (백색 장방형) 당뇨 혈당 강하제
			\end{itemize}
			\end{block}			


								
		\end{frame}	 % ---------------------------------------------					 


	%	----------------------------------------------------------  
	%		Frame   
	%	---------------------------------------------------------- page 20	저녁약
		\begin{frame} [t,plain]

			\begin{block} { 저녁약 : 메리놀병원 - 제일약국 (이비인후과) }
			\begin{itemize}
			\item 	[1] 케이컵정 50mg  위산분비억제제
			\item 	[2] 알레스틴 정   항알레르기약
			\item 	[3] 에르도스 캅셀 300 mg 객담 배출
			\item 	[4] 아세펙트캡슐 기관지 확장제
			\end{itemize}
			\end{block}			

			\begin{block} { 저녁약 : 메리놀병원 - 소망약국 (비뇨의학과)}
			\begin{itemize}
			\item 	한림 피나스테리드정 (청색 원형)
			\item 	일양하이트린정 (주황색 원형)
			\end{itemize}
			\end{block}			
								
		\end{frame}	 % ---------------------------------------------					 




%	%	----------------------------------------------------------  		도서관
%	%		Frame   
%	%	---------------------------------------------------------- page 21	강서도서관
%		\begin{frame} [t,plain]
%
%			\begin{block} { 도서관 : 강서 도서관 }
%
%			\begin{itemize}
%			\item 	2022/04/03  일 ~ 2022/04/17 일
%			\item 	[1.] 도교대 수재들의 리얼 종이접기
%			\item 	[2.] 불교 종이 접기 
%			\item 	[3.] (원욱 스님의) 나를 바꾸는 화엄경
%			\item 	[4.] Hello 실전 파이썬 프로그래밍 world
%			\item 	[5.] 일잘러는 노션으로 일합니다
%			\end{itemize}
%
%
%			\end{block}			
%								
%		\end{frame}	 % ---------------------------------------------					 


%	%	----------------------------------------------------------  		도서관
%	%		Frame   
%	%	---------------------------------------------------------- page 22	사하도서관
%		\begin{frame} [t,plain]
%
%			\begin{block} { 도서관 : 사하 도서관 }
%
%			\begin{itemize}
%			\item 	2022/03/19 토 ~ 2022/04/02 토
%			\item 	[1.] 파이썬 프로그래밍
%			\item 	[2.] 피그마 완벽 활용서
%			\item 	[3.] 소설 공자
%			\item 	[4.] 반체제 인사의 리더에서 성인 되기까지 우리가 몰랐던 공자 이야기
%			\item 	[5.] 애쓰지 않기 위해 노력하기 
%			\end{itemize}
%
%			\end{block}			
%								
%		\end{frame}	 % ---------------------------------------------					 


%	%	----------------------------------------------------------  		도서관
%	%		Frame   
%	%	---------------------------------------------------------- page 23 	구덕도서관
%		\begin{frame} [t,plain]
%
%			\begin{block} { 도서관 : 구덕 도서관 }
%
%			\begin{itemize}
%			\item 	2022/07/07  목 ~ 2022/07/21 목
%			\item 	[1.] Git 교과서
%			\end{itemize}
%
%
%
%			\end{block}			
%								
%		\end{frame}	 % ---------------------------------------------					 


%	%	----------------------------------------------------------  		도서관
%	%		Frame   
%	%	---------------------------------------------------------- page 24	중앙 도서관
%		\begin{frame} [t,plain]
%
%			\begin{block} { 도서관 : 중앙 도서관 }
%
%
%			\begin{itemize}
%			\item 	2022/04/02  토 ~ 2022/04/16 토
%			\item 	[1.] 킨포크 가든
%			\item 	[2.] 정원의 세계
%			\item 	[3.] 마애불을 찾아가는 여행 
%			\item 	[4.] 초발심 자경문 및 치문
%			\item 	[5.] (무비스님의) 초발심자경문 강설
%			\end{itemize}
%
%			\end{block}			
%								
%		\end{frame}	 % ---------------------------------------------					 
%


	%	----------------------------------------------------------  		도서관
	%		Frame   
	%	---------------------------------------------------------- page 25	수정 도서관
		\begin{frame} [t,plain]

			\begin{block} { 도서관 : 수정 도서관 }


			\begin{itemize}
			\item 	2022/07/23 토 ~ 2022/08/06 토
			\item 	[06]    허리 좀 펴고 삽시다    
		      \item 	[07]    그러니 바람아 불기만 하지말고 이루어져라
		      \item 	[08]    실무에서 바로 쓰는 파워셀
		      \item 	[09]    한국의 범종
		      \item 	[10]    Git 교과서
			\end{itemize}




			\end{block}			
								
		\end{frame}	 % ---------------------------------------------					 


%	%	----------------------------------------------------------  		도서관
%	%		Frame   
%	%	---------------------------------------------------------- page 26	반여 도서관
%		\begin{frame} [t,plain]
%
%			\begin{block} { 도서관 : 반여 도서관 }
%
%
%			\begin{itemize}
%			\item 	2022/03/13 일 - 2022/03/27 일
%			\item 	[1.] 루스 베네딕트 국화와 칼
%			\item 	[2.] 벌의 사생활
%			\item 	[3.] 까칠한 조땡의 인포그래픽 디자인 with 24 cases
%			\item 	[4.] 이것이 인포그래픽이다
%			\item 	[5.] UX 디자인의 모든것
%			\end{itemize}
%
%			\end{block}			
%								
%		\end{frame}	 % ---------------------------------------------					 



%	%	----------------------------------------------------------  		도서관
%	%		Frame   
%	%	---------------------------------------------------------- page 27	기장 도서관
%		\begin{frame} [t,plain]
%
%			\begin{block} { 도서관 : 기장 도서관 }
%
%			\begin{itemize}
%			\item 	2022/06/24 금 ~ 2022/07/08 금
%			\item 	[06] 내 음악 만들기
%			\item 	[07] 뻔뻔한 작곡법
%			\item 	[08] 8주 완성 작곡법
%			\item 	[09] 최신 가요 백과
%			\item 	[10] 세계명가곡집
%			\end{itemize}
%			\end{block}			
%								
%		\end{frame}	 % ---------------------------------------------					 

%	%	----------------------------------------------------------  		도서관
%	%		Frame   
%	%	---------------------------------------------------------- page 27	기장 도서관
%		\begin{frame} [t,plain]
%
%			\begin{block} { 도서관 : 기장 도서관 }
%
%			\begin{itemize}
%			\item 	2022/06/24 금 ~ 2022/07/08 금
%			\item 	[11] 음악이론과 반주법의 실제
%			\item 	[12] 하모니카 교실
%			\item 	[13] 오디오 프로세싱
%			\item 	[14] 쓰담쓰담 갈리바
%			\item 	[15] 엄지 피아노 갈리바
%			\end{itemize}
%			\end{block}			
%								
%		\end{frame}	 % ---------------------------------------------					 


%	%	----------------------------------------------------------  		도서관
%	%		Frame   
%	%	---------------------------------------------------------- page 27	내리 새라 도서관
%		\begin{frame} [t,plain]
%
%			\begin{block} { 도서관 : 내리 새라 도서관 }
%
%			\begin{itemize}
%			\item 	2022/06/24 금 ~ 2022/07/08 금
%			\item 	[16] 알기 쉽게 배우는 3D프린터
%			\item 	[17] 오늘도 나무를 그린다
%			\item 	[18] 최재천의 공부
%			\item 	[19] 이기는 습관
%			\item 	[20] 다시 미분적분
%			\end{itemize}
%
%			\end{block}			
%								
%		\end{frame}	 % ---------------------------------------------					 


%	%	----------------------------------------------------------  		도서관
%	%		Frame   
%	%	---------------------------------------------------------- page 29	해운대 도서관
%		\begin{frame} [t,plain]
%
%			\begin{block} { 도서관 : 해운대 도서관 }
%
%			\begin{itemize}
%			\item 	2022/04/29 금 ~ 2022/05/13 금
%			\item 	[1.] (도표로 읽는) 불교 입문
%			\item 	[2.] 도표로 읽는 불교 교리
%			\item 	[3.] 도해 금강경 : 그림과 도표로 읽는 견고하고 단단한 반야 지혜의 총체
%			\item 	[4.] (도표로 읽는) 경전 입문 : 방대한 팔만대장경의 세계가 도표로 한눈에 들어온다
%			\item 	[5.] 거꾸로 읽는 그리스 로마사 : 신화가 아닌 보통 사람의 삶으로 본 그리스 로마 시대
%			\end{itemize}
%
%			\end{block}			
%								
%		\end{frame}	 % ---------------------------------------------					 



%%	----------------------------------------------------------  		도서관
%%		Frame   
%%	---------------------------------------------------------- page 30	남구도서관
%		\begin{frame} [t,plain]
%
%			\begin{block} { 도서관 : 남구 도서관 }
%
%			\begin{itemize}
%			\item 	2022/05/13 금 ~ 2022/05/27 금
%			\item 	[1.] 영상 촬영 편집 스킬업
%			\item 	[2.] 일자러는 노션으로 일합니다
%			\item 	[3.] 자네 좌뇌에게 속았네
%			\item 	[4.] 서성제 : 괴로움과 괴로움의 소멸
%			\item 	[5.] 하마터면 깨달을뿐리를 그림으로 쉽게 풀이한)천문학 사전
%			\end{itemize}
%
%			\end{block}			
%								
%		\end{frame}	 % ---------------------------------------------					 
%




%	%	----------------------------------------------------------  		도서관
%	%		Frame   
%	%	---------------------------------------------------------- page 31	해운대 인문학 도서관
%		\begin{frame} [t,plain]
%
%			\begin{block} { 도서관 : 해운대 인문학 도서관 }
%
%			\begin{itemize}
%			\item 	2022/03/13 일 - 2022/03/27 일
%			\item 	[1.] 한국춘란 가이드북 : 전문가편
%			\item 	[2.] 과거보고 벼슬하고 : 관리의 길
%			\item 	[3.] 365수학
%			\item 	[4.] 한국춘란 가이드북 : 입문편
%			\item 	[5.] AI는 차별을 인간에게서 배운다
%			\end{itemize}
%
%
%			\end{block}			
%								
%		\end{frame}	 % ---------------------------------------------					 
%

	%	----------------------------------------------------------  		도서관
	%		Frame   
	%	---------------------------------------------------------- page 32	금정 도서관
		\begin{frame} [t,plain]

			\begin{block} { 25 도서관 : 금정 도서관 }

			\begin{itemize}
			\item 	2022/07/23 토 ~ 2022/08/06 토
			\item 	[11]     맛지마 니까야 3
		      \item 	[12]     맛지마 니까여 4
		      \item 	[13]     상윳다 니까야  2
		      \item 	[14]     상윳다 니까야  3  
		      \item 	[15]     상윳다 니까야  5
			\end{itemize}

			\end{block}			
								
		\end{frame}	 % ---------------------------------------------					 


	%	----------------------------------------------------------  		도서관
	%		Frame   
	%	---------------------------------------------------------- page 26	빈 페이지
		\begin{frame} [t,plain]

								
		\end{frame}	 % ---------------------------------------------					 


%	%	----------------------------------------------------------  		도서관
%	%		Frame   
%	%	---------------------------------------------------------- page 26	빈 페이지
%		\begin{frame} [t,plain]
%
%								
%		\end{frame}	 % ---------------------------------------------					 


%	%	----------------------------------------------------------  		도서관
%	%		Frame   
%	%	---------------------------------------------------------- page 24	구덕 도서관
%		\begin{frame} [t,plain]
%
%			\begin{block} { 도서관 : 구덕 도서관 }
%
%
%			\begin{itemize}
%			\item 	2022/06/14 화 ~ 2022/06/28 화
%			\item 	[1.] 첫 통계 with 베이즈
%			\item 	[2.] 쓸모있는 음악책
%			\item 	[3.] 식물의 세계사
%			\item 	[4.] 경이로운 우주의 역사
%			\item 	[5.] 한자의 기원
%			\end{itemize}
%
%			\end{block}			
%								
%		\end{frame}	 % ---------------------------------------------					 




	%	----------------------------------------------------------  		도서관
	%		Frame   
	%	---------------------------------------------------------- page 26	빈 페이지
		\begin{frame} [t,plain]

								
		\end{frame}	 % ---------------------------------------------					 



	%	----------------------------------------------------------  		도서관
	%		Frame   
	%	---------------------------------------------------------- page 26	빈 페이지
		\begin{frame} [t,plain]

								
		\end{frame}	 % ---------------------------------------------					 


	%	----------------------------------------------------------  		도서관
	%		Frame   
	%	---------------------------------------------------------- page 26	빈 페이지
		\begin{frame} [t,plain]

								
		\end{frame}	 % ---------------------------------------------					 


%	%	----------------------------------------------------------  		도서관
%	%		Frame   
%	%	---------------------------------------------------------- page 33	부전도서관
%		\begin{frame} [t,plain]
%
%			\begin{block} { 26 도서관 : 부전 도서관 }
%
%			\begin{itemize}
%			\item 	2022/2/08 화 - 2022/02/22 화
%			\item 	[1.] 응용역학=Hand-written applied mechanics
%			\item 	[2.] 문화 속의 수학
%			\item 	[3.] 비숲 : 긴팔원숭이 박사의 밀림 모험기
%			\item 	[4.] (파이썬으로 다시 배우는) 핵심 고등 수학 : 수포자 프로그래머를 위한 손에 잡히는 기초 수학
%			\item 	[5.] 나는 식물을 따라 걷기로 했다
%			\end{itemize}
%
%			\end{block}			
%								
%		\end{frame}	 % ---------------------------------------------					 




	%	----------------------------------------------------------  		도서관
	%		Frame   
	%	---------------------------------------------------------- page 34	옹기종기 도서관	
		\begin{frame} [t,plain]

			\begin{block} { 27  도서관 : 옹기종기 도서관 }

			\begin{itemize}
			\item 	2022/07/22 금 ~ 2022/08/05 금
			\item 	[01]    최초의 여신 인안나
		      \item 	[02]    수행자의 정원
		      \item 	[03]   처음읽는 우파니샤드
		      \item 	[04]   베다 인류 최초의 거룩한 가르침
		      \item 	[05]   생활속의 바그바드 기타
			\end{itemize}



			\end{block}			
								
		\end{frame}	 % ---------------------------------------------					 


	%	----------------------------------------------------------  		군법당
	%		Frame   
	%	---------------------------------------------------------- page 28
		\begin{frame} [t,plain]

			\begin{block} { page 28 : 청파팀 군법당 }

			\begin{itemize}
			\item 	나무 젓가락 : 혜웅사
			\item 	걸이용 핀 : 혜웅사
			\item 	라면 포트 (2,7대대 개금 백양사)
			\item 	냄비 받침대 : 3대대 라면 먹을 때 받침대
			\end{itemize}



			\end{block}			
								
		\end{frame}	 % ---------------------------------------------					 


%			\item 	1주차
%			\item 	태종대 여단,1 대대	: 보월 김형식, 고불행 박영애, 보조 김대희
%			\item 	개금 2,7 대대 	: 보운 천유실, 법륜행 강영옥, 법은회 김미리 
%			\item 	송도 4 대대 	: 보현행 최미경, 여은심 윤점분
%
%			\item 	2주차
%			\item 	태종대 여단,1 대대	: 본각 김영신, 환희행 김은희, 보운 천유실
%			\item 	개금 2,7 대대 	: 불심 강만석, 정진화 박순남, 보광 이희규
%			\item 	송도 4 대대 	: 연담 강동섭, 향청 권미애, 행원 이종찬
%
%			\item 	3주차
%			\item 	태종대 여단,1 대대	: 월인 김필근, 수성행 황남숙, 수경도 정미숙
%			\item 	개금 2,7 대대 	: 현소 안병렬, 현중지 정남선,  정연 장성대
%			\item 	송도 4 대대 	: 해공자 안애영, 법운화 윤경숙, 보조 김대희 

%	%	----------------------------------------------------------  
%	%		Frame   
%	%	---------------------------------------------------------- page 26
%		\begin{frame} [t,plain]
%
%			\begin{block} { page 26 }
%
%			\end{block}			
%								
%		\end{frame}	 % ---------------------------------------------					 


%	%	----------------------------------------------------------  
%	%		Frame   
%	%	---------------------------------------------------------- page 27
%		\begin{frame} [t,plain]
%
%			\begin{block} { page 27 }
%
%			\end{block}			
%								
%		\end{frame}	 % ---------------------------------------------					 


%	%	----------------------------------------------------------  
%	%		Frame   
%	%	---------------------------------------------------------- page 28
%		\begin{frame} [t,plain]
%
%			\begin{block} { page 28 }
%
%			\end{block}			
%								
%		\end{frame}	 % ---------------------------------------------					 


	%	----------------------------------------------------------  		도서관일자
	%		Frame   
	%	---------------------------------------------------------- page 29
		\begin{frame} [t,plain]

			\begin{block} { page 29 : 도서관 }
			\setlength{\leftmargini}{3em}			

			\begin{itemize}
%			\item 	기장 : 06. 24. (금) - 07. 08. (금)
%			\item 	정관 : 06. 24. (금) - 07. 08. (금)
%			\item 	중앙 : 05. 30. (월) - 06. 14. (화)
			\item 	수정 : 7. 23. (토) - 8. 06. (토)
%			\item 	반여 : 03. 13. (일) - 03. 27. (일)
%			\item 	사상 : 12. 28. (화) - 1. 11. (화)
%			\item 	구덕 : 07. 07. (목) - 07. 21. (목)
%			\item 	강서 : 04. 03. (일) - 04  17. (일)
			\item 	금정 : 07. 23. (토) - 08. 06. (토)
%			\item 	남구 : 1. 7. (금) - 1. 21. (금)
%			\item 	해운대 : 5. 27. (금) - 6. 10. (금)
%			\item 	해운대 인문학 : 3. 13. (일) - 3. 27. (일)
%			\item 	내리새라 : 4. 15. (금) - 4. 29. (금)
			\item 	옹기종기 : 7. 22. (금) - 8. 05. (금)
%			\item 	사하 : 3. 19. (토) - 4. 02. (토)
%			\item 	부전 : 2. 08. (화) - 2. 22. (화)
			\end{itemize}

			\end{block}			


		\end{frame}	 % ---------------------------------------------					 


	%	----------------------------------------------------------  
	%		Frame   
	%	---------------------------------------------------------- page 30
		\begin{frame} [t,plain]

			\begin{block} { page 30 : 능인회  }

			\begin{itemize}
			\item 	
			\item 	
			\item 	임원 회비 50.  연간 회비 120.  합계 170.
			\end{itemize}
			\end{block}			

								
		\end{frame}	 % ---------------------------------------------					 





	%	----------------------------------------------------------  차량 운행
	%		Frame
	%	---------------------------------------------------------- page 31
		\begin{frame} [t,plain]

			\begin{block} { 차량 운행 }
			\setlength{\leftmargini}{3em}			
%		\begin{huge}
		\begin{large}

			\begin{itemize}
			\item 	시작 
			\item 	
			\item 	종료 
			\item 	
			\item 	
			\item 	
			\end{itemize}
		\end{large}
%		\end{huge}

			\end{block}			
								
		\end{frame}	 % ---------------------------------------------					 


	%	----------------------------------------------------------  찾을것
	%		Frame   찾을것
	%	---------------------------------------------------------- page 32
		\begin{frame} [t,plain]

			\begin{block} { 찾을것 }
			\setlength{\leftmargini}{3em}			

			\begin{itemize}
			\item 철자 	
			\item 	
			\item 	
			\end{itemize}

			\end{block}			


			\begin{block} { 가져올것 }
			\setlength{\leftmargini}{3em}			

			\begin{itemize}
%			\item 보온병 큰것
			\item 	
			\item 	
			\end{itemize}

			\end{block}			

								
		\end{frame}	 % ---------------------------------------------					 




% ------------------------------------------------------------------------------
% End document
% ------------------------------------------------------------------------------




\end{document}



초량 교차로
제1지하차도 교차로
감만 사거리
감만 현대 아파트 사거리
동명 오거리
송정어귀 삼거리
송정 삼거리
연화육교 교차로
청강 사거리
교리 삼거리
일광 교차로
화전 교차로
좌천 삼거리
좌천 사거리
좌동 삼거리 ( 동남권 원자력의학원)
장안IC 교차로
용소 삼거리
기룡 교